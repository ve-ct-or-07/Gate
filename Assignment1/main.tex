\let\negmedspace\undefined
\let\negthickspace\undefined
\documentclass[journal]{IEEEtran}
\usepackage[a5paper, margin=10mm, onecolumn]{geometry}
%\usepackage{lmodern} % Ensure lmodern is loaded for pdflatex
\usepackage{tfrupee} % Include tfrupee package

\setlength{\headheight}{1cm} % Set the height of the header box
\setlength{\headsep}{0mm}     % Set the distance between the header box and the top of the text

\usepackage{gvv-book}
\usepackage{gvv}
\usepackage{cite}
\usepackage{amsmath,amssymb,amsfonts,amsthm}
\usepackage{algorithmic}
\usepackage{graphicx}
\usepackage{textcomp}
\usepackage{xcolor}
\usepackage{txfonts}
\usepackage{listings}
\usepackage{enumitem}
\usepackage{mathtools}
\usepackage{gensymb}
\usepackage{comment}
\usepackage[breaklinks=true]{hyperref}
\usepackage{tkz-euclide} 
\usepackage{listings}
% \usepackage{gvv}                                        
\def\inputGnumericTable{}                                 
\usepackage[latin1]{inputenc}                                
\usepackage{color}                                            
\usepackage{array}                                            
\usepackage{longtable}                                       
\usepackage{calc}                                             
\usepackage{multirow}                                         
\usepackage{hhline}                                           
\usepackage{ifthen}                                           
\usepackage{lscape}
\usepackage{multicol}
\begin{document}

\bibliographystyle{IEEEtran}
\vspace{3cm}

\title{2007\\AE : Aerospace Engineering}
\author{AI24BTECH11022 - Pabbuleti Venkata Charan Teja}
\maketitle

\renewcommand{\thefigure}{\theenumi}
\renewcommand{\thetable}{\theenumi}


\begin{enumerate}
\item Which one of the following engines should be used by a subsonic passenger transport airplane for minimum specific fuel consumption?
\begin{multicols}{2}
\begin{enumerate}
\item Turbojet engine with afterburner
\item Turbofan engine 
\item Ramjet engine
\item Scramjet engine
\end{enumerate}
\end{multicols}


\item A spring-mass-damper system with a mass of $1kg$ is found to have a damping ratio of $0.2$ and a natural frequency of $5rad/s$. The damping of the system is given by
\begin{multicols}{2}
\begin{enumerate}
\item $2Ns/m$
\item $2N/s$
\item $0.2kg/s$
\item $0.2N/s$
\end{enumerate}
\end{multicols}


\item If $f\brak{\theta}=\myvec{\cos{\theta}&\sin{\theta}\\ -\sin{\theta}&\cos{\theta}}$, then $f\brak{\alpha}f\brak{\beta}=$
\begin{multicols}{2}
\begin{enumerate}
\item $f\brak{\frac{\alpha}{\beta}}$
\item $f\brak{\alpha+\beta}$
\item $f\brak{\alpha-\beta}$
\item $2\times 2$ zero matrix 
\end{enumerate}
\end{multicols}


\item An artificial satellite remains in orbit and does not fall to the earth because
\begin{enumerate}
\item the centrifugal force acting on it balances the gravitational attraction 
\item the on-board rocket motors provide continuous boost to keep it in orbit
\item its transverse velocity keeps it from hitting the earth although it falls continuously 
\item due to its high speed it derives sufficient lift from the rarefied atmosphere
\end{enumerate}


\item The Euler iteration formula for numerically integrating a first order nonlinear differential equation of the form $x=f\brak{x}$, with a constant step size of $\Delta t$ is
\begin{multicols}{2}
\begin{enumerate}
\item $x_{k+1}=x_{k}-\Delta t\times f\brak{x_{k}}$
\item $x_{k+1}=x_{k}-\frac{\Delta t^{2}}{2}\times f\brak{x_{k}}$
\item $x_{k+1}=x_{k}-\frac{1}{\Delta t}\times f\brak{x_{k}}$
\item $x_{k+1}=x_{k}+\Delta t\times f\brak{x_{k}}$
\end{enumerate}
\end{multicols}


\item The number of natural frequencies of an elastic beam with cantilever boundary conditions is
\begin{multicols}{2}
\begin{enumerate}
\item $1$
\item $3$
\item $1000$
\item Infinite
\end{enumerate}
\end{multicols}


\item For maximum range of a glider, which of the following conditions is true?
\begin{multicols}{2}
\begin{enumerate}
\item lift to drag ratio is maximum
\item rate of descent is minimum
\item descent angle is maximum
\item lift to weight ratio is maximum
\end{enumerate}
\end{multicols}


\item An airplane with a larger wing as compared to a smaller wing will necessarily have
\begin{enumerate}
\item more longitudinal static stability
\item less longitudinal static stability
\item same longitudinal static stability
\item more longitudinal static stability for an aft tail airplane if aerodynamic center of the larger wing is behind the center of gravity of the airplane
\end{enumerate}


\item The minimum value of $f\brak{x}=x^{2}-7x+30$ occurs at
\begin{multicols}{2}
\begin{enumerate}
\item $x=\frac{7}{2}$
\item $x=\frac{7}{30}$
\item $x=\frac{30}{7}$
\item $x=30$
\end{enumerate}
\end{multicols}


\item Two airplanes are identical except for the location of the wing. The longitudinal static stability of the airplane with low wing configuration will be
\begin{enumerate}
\item more than the airplane with high wing configuration
\item less than the airplane with high wing configuration
\item same as the airplane with high wing configuration
\item more if elevator is deflected
\end{enumerate}

\item For a fixed center of gravity location of an airplane, when the propeller is mounted on the nose of the fuselage
\begin{enumerate}
\item longitudinal static stability increases
\item longitudinal static stability decreases
\item longitudinal static stability remains same
\item longitudinal static stability is maximum
\end{enumerate}


\item Let an airplane in a steady level flight be trimmed at a certain speed. A level and steady flight at a higher speed could be achieved by changing
\begin{multicols}{2}
\begin{enumerate}
\item engine throttle only
\item elevator only
\item throttle and elevator together
\item rudder only
\end{enumerate}
\end{multicols}


\item For a plane strain problem in the $x-y$ plane, in general, the non-zero stress terms are
\begin{multicols}{2}
\begin{enumerate}
\item $\sigma_{zz},\sigma_{xz},\sigma_{yz},\sigma_{xy}$
\item $\sigma_{zz},\sigma_{xz},\sigma_{yz},\sigma_{xy}$
\item $\sigma_{xx},\sigma_{xy},\sigma_{yy},\sigma_{xz}$
\item $\sigma_{xx},\sigma_{yy},\sigma_{xy},\sigma_{zz}$
\end{enumerate}
\end{multicols}


\item For an elastic anisotropic solid, the number of independent elastic constants in its constitutive equations is
\begin{multicols}{2}
\begin{enumerate}
\item $2$
\item $9$
\item $21$
\item $36$
\end{enumerate}
\end{multicols}


\item Total pressure at a point is defined as the pressure when the flow is brought to rest
\begin{multicols}{2}
\begin{enumerate}
\item adiabatically
\item isentropically
\item isothermally
\item isobarically
\end{enumerate}
\end{multicols}


\item The drag divergence Mach number of an airfoil
\begin{enumerate}
\item is a fixed number for a given airfoil
\item is always higher than the critical Mach number
\item is equal to the critical Mach number at zero angle of attack
\item is the Mach number at which a shock wave first appears on the airfoil
\end{enumerate}


\item On which one of the following thermodynamic cycles does an ideal ramjet operate?
\begin{multicols}{2}
\begin{enumerate}
\item The Rankine cycle
\item The Brayton cycle
\item The Camot cycle
\item The Otto cycle
\end{enumerate}
\end{multicols}
\end{enumerate}
\end{document}